\documentclass{rapportCS}
\usepackage{lipsum}

\usepackage{tikz}
\usetikzlibrary{babel,positioning,shapes}
\usepackage[utf8]{inputenc}
\usepackage[T1]{fontenc}
\usepackage{mathtools}
\usepackage[thinc]{esdiff}
\usetikzlibrary{babel,positioning,shapes}
\usepackage{float}    % For tables and other floats
\usepackage{verbatim} % For comments and other
\usepackage{amsmath}  % For math
\usepackage{amssymb}  % For more math
\usepackage{amsthm}
\usepackage{amsfonts}
\usepackage{amssymb}
\usepackage{mathrsfs}
\usepackage{graphicx}
\renewcommand{\qedsymbol}{$\blacksquare$}
\renewcommand{\thesection}{\Roman{section}} 
\usepackage[shortlabels]{enumitem}
\usepackage{listings}
\usepackage{xcolor}
\usepackage{caption}
\usepackage{lastpage}

\definecolor{codegreen}{rgb}{0,0.6,0}
\definecolor{codegray}{rgb}{0.5,0.5,0.5}
\definecolor{codepurple}{rgb}{0.58,0,0.82}
\definecolor{backcolour}{rgb}{0.95,0.95,0.92}

\lstdefinestyle{mystyle}{
    backgroundcolor=\color{backcolour},   
    commentstyle=\color{codegreen},
    keywordstyle=\color{magenta},
    numberstyle=\tiny\color{codegray},
    stringstyle=\color{codepurple},
    basicstyle=\ttfamily\footnotesize,
    breakatwhitespace=false,         
    breaklines=true,                 
    captionpos=b,                    
    keepspaces=true,                 
    numbers=left,                    
    numbersep=5pt,                  
    showspaces=false,                
    showstringspaces=false,
    showtabs=false,                  
    tabsize=2
}

\lstset{style=mystyle}



\title{Rapport ECS - Machine learning} %Titre du fichier

\begin{document}

%----------- Informations du rapport ---------

\titre{{\LARGE \bfseries Projej\\
[0.1cm] }
{\large \bfseries AirBnB Berlin price \\[0.1cm] }} %Titre du fichier .pdf
\UE{Apprentissage automatique}

\enseignant{Enseignant}%Nom de l'enseignant


\eleves{ 
José Lucas \textsc{De Melo Costa}
Victor Felipe \textsc{Domingues do Amaral}
} %Nom des élèves

%----------- Initialisation -------------------
        
\fairemarges %Afficher les marges
\fairepagedegarde %Créer la page de garde


%====================== INCLUSION DES PARTIES ======================
%régler l'espacement entre les lignes
\newcommand{\HRule}{\rule{\linewidth}{0.5mm}}

%recommencer la numérotation des pages à "1"

%récupérer les citation avec "/footnotemark"


%choix du style de la biblio
%\bibliographystyle{plain}
%inclusion de la biblio
%\bibliography{bibliographie}
%voir wiki pour plus d'information sur la syntaxe des entrées d'une bibliographie


\end{document}
